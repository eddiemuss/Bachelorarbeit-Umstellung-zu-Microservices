\section*{Abstract}

%Problemstellung
Mit wachsender Codebase erhöht sich der Aufwand, der notwendig ist, um neue Funktionen zu entwickeln und zu implementieren. Dies liegt besonders daran, dass sich im Laufe der Entwicklung viele Abhängigkeiten zwischen Klassen und Methoden entwickelt haben, die eine schnelle Weiterentwicklung hemmen. Um dem entgegen zu wirken, hat PluraPolit beschlossen, die aktuelle monolithische Softwarearchitektur in eine Microservice-Architektur umzustellen.


%Zielsetzung
Das Ziel dieser Abhandlung ist es, für PluraPolit die Bedingungen zu ermitteln, die für eine mögliche Umstellung erforderlich sind und eine klare Beurteilung für das Vorhaben abzugeben.

%Methodik
Um dies herauszufinden, wurden Bedingungen aus einer Literaturrecherche erstellt und Leitfadeninterviews mit Experten durchgeführt. Dabei wurden die Fragen für die Interviews aus den Bedingungen abgeleitet. Die Befragten wurden aufgrund ihrer jahrelangen Erfahrung im Bereich der Microservice-Architektur und Start-up-Branche ausgewählt.

%Ergebnisse
Die Ergebnisse zeigten, dass vor der Umstellung folgende Bedingungen erfüllt sein müssen:
\begin{enumerate}
	\item Die Umstellung ist für ein Start-up lukrativ
	\item Das System weist eine gewisse Komplexität auf
	\item Die Geschäftsabläufe lassen sich voneinander trennen
	\item Es ist möglich, die Verantwortung für ein Geschäftsprozess an ein eigenständiges Team zu geben
	\item Die Services können untereinander über Schnittstellen kommunizieren
	\item Das Netzwerk ermöglicht die Kommunikation zwischen Services
	\item Das Netzwerk ist vor Zugriffen von Ungefugten gesichert
\end{enumerate}

Als wichtigste Bedingung stellte sich die Bewertung der Profitabilität heraus, welche vom zu erwartenden Mehrwert und den notwendigen Kosten abhängt.

%Bezug zu PluraPolit
Da für PluraPolit keine nachvollziehbaren Vorteile aus einer Umstellung erkannt werden, während Kosten durch weitere Schulungen oder externe Fachkräfte absehbar sind, ist eine Umstellung für PluraPolit zum aktuellen Zeitpunkt nicht zu empfehlen.

\newpage

\section*{Abstract (Englisch)}

%Problemstellung
As the codebase grows, the effort required to develop and implement new functions increases. This is especially due to the fact that during development many dependencies between classes and methods have evolved. To counter this, PluraPolit has decided to convert the current monolithic architecture into a microservice architecture.

%Zielsetzung
The aim of this paper is to identify the conditions necessary for PluraPolit for a possible conversion and to provide a clear evaluation for the project.

%Methodik
To research on this topic, conditions were derived from a literature search and guideline interviews with experts were conducted. The questions for the interviews were derived from the conditions. The interviewees were selected on the basis of their years of experience in the field of microservice architecture and start-up industries.

%Ergebnisse
The results show that the following conditions must be met:
\begin{enumerate}
	\item The conversion need to be lucrative for a start-up
	\item The system has a certain degree of complexity
	\item The business processes are able to be separated
	\item It is possible to transfer the responsibility for a business process to an independent team
	\item The services can communicate with each other via interfaces
	\item The network enables communication between services
	\item The network is secured against unauthorized access
\end{enumerate}

The most important condition was found to be profitability, which depends on the expected added value and the necessary costs.

%Bezug zu PluraPolit
Since no comprehensible benefits from a changeover have been identified, while costs for further training or external specialists are foreseeable, a changeover for PluraPolit is not recommended at this time.
