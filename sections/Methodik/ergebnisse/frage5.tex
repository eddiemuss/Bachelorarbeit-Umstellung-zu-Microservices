\subsubsection{Frage 5}
\label{sec:frage5}

\textit{Mit Ihrem aktuellen Wissensstand, welche Softwarearchitektur empfehlen Sie PluraPolit?}

Alle drei Experten empfolen einen Monolithen (siehe \combref{appendix:t-36}, \combref{appendix:t-38} und \combref{appendix:s-31}).

Alexander Troppman und Sebastian Schlaak haben weiterführend geraten, das Backend und das Frontend voneinander zu trennen. Herr Schlaaks Vorschlag war es, im Frontend ein \textit{\enquote{modernes JavaScript Framework}} zu verwenden und das \textit{\enquote{Backoffice}} der Mitarbeiter von der Darstellung der Endkunden zu trennen (siehe \combref{appendix:s-36}). Beide würden den Ansatz wählen, um die Flexibilität zu haben das Frontend bzw. das Backend auszutauschen.

Sowohl Herr Rahles, als auch Herr Troppman gaben an, dass sie einen Monolithen empfehlen, da es keine separate Geschäftsdomäne gibt, die eine Aufteilung in Services rechtfertigt.\footnote{
\textit{\enquote{Ich glaube, dass es keine separate Business Domain gibt, wo man sagt, die muss zwangsläufig ausgelagert werden.}} (siehe \combref{appendix:r-40}); \textit{\enquote{Ich sehe da vom fachlichen her nichts, was eine Microservicearchitektur rechtfertigt.}} (siehe \combref{appendix:t-37})
}

Des Weiteren stand Herr Rahles der Umstellung zu Microservices ablehnend gegenüber, da nicht genügend Entwickler vorhanden sind (siehe \combref{appendix:r-39}).