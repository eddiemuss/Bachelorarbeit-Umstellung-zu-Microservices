\subsubsection{Frage 2}

\textit{Microservices ermöglicht es Team unabhängig von einander an unterschiedlichen Services zu arbeiten. Welche Rahmen Bedingungen sehen Sie als notwendig, dass Teams separat voneinander Arbeiten können?}

%Todo Kommasetzung in der Frage überprüfen

Herr Rahles fügte den Vermerk hinzu, dass die Teams stets zu einem Unternehmen gehören und nie ganz unabhängig sind.\footnote{
\textit{\enquote{Auch diese Teams gehören zu einem Unternehmen […], das heißt unabhängig voneinander sind sie nie.}} (siehe \combref{appendix:r-15})
}

Herr Schlaak und Herr Troppman führten an, dass insbesondere Schnittstellen definiert und beschrieben sein müssen.\footnote{
\textit{\enquote{Ganz wichtig ist, dass die Schnittstellen der Services entsprechend gut beschrieben sind […]}} (siehe \combref{appendix:s-13}); \textit{\enquote{[…] die Teams müssen sich einig sein, über welche Schnittstelle die Services kommunizieren.}} (siehe \combref{appendix:t-19}); \textit{\enquote{Schnittstellen nach außen müssen geklärt sein.}} ( siehe \combref{appendix:t-22})
}

Weiterführend wurde von Herr Troppman eine Struktur hervorgehoben, die Abläufe genauer beschreibt.\footnote{
\textit{\enquote{[…] zumindest benötigt man Strukturen, sodass man weiß, was wie abläuft.}} (siehe \combref{appendix:t-22})
} Diesen Ansatz erwähnte auch Herr Rahles, als er die Notwendigkeit von Projektmanagement ansprach.\footnote{
\textit{\enquote{[…] es ist einfach sinnvoll zu gucken, dass […] die Teams entsprechend des Projektmanagements richtig aufgestellt sind.}} (siehe \combref{appendix:r-22})
} Daraus lässt sich folgern, dass Struktur notwendig ist, damit Teams separat arbeiten können.
