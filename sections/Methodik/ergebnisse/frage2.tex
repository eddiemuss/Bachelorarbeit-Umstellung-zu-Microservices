\subsubsection{Frage 2}
\label{sec:frage2}

\textit{Microservices ermöglichen es Teams, unabhängig voneinander an unterschiedlichen Services zu arbeiten. Welche Rahmenbedingungen sehen Sie als notwendig, dass Teams separat voneinander arbeiten können?}

Herr Rahles fügte den Vermerk hinzu, dass die Teams stets zu einem Unternehmen gehören und nie ganz unabhängig sind.\footnote{
\textit{\enquote{Auch diese Teams gehören zu einem Unternehmen […], das heißt unabhängig voneinander sind sie nie.}} (siehe \combref{appendix:r-15})
}
Auch beurteilte er, dass Teams \textit{\enquote{interdisziplinär aufgestellt sein [müssen], [damit] sie wirklich unabhängig voneinander arbeiten können}} (siehe \combref{appendix:r-16}) und wies in seinem Interview daraufhin, dass Teams \textit{\enquote{innerhalb [ihrer] Business Domäne [...] der Owner sein [müssen]}} (siehe \combref{appendix:r-20}).

Herr Schlaak und Herr Troppmann führten an, dass insbesondere Schnittstellen definiert und beschrieben sein müssen.\footnote{
\textit{\enquote{Ganz wichtig ist, dass die Schnittstellen der Services entsprechend gut beschrieben sind […]}} (siehe \combref{appendix:s-13}); \textit{\enquote{[…] die Teams müssen sich einig sein, über welche Schnittstelle die Services kommunizieren.}} (siehe \combref{appendix:t-19}); \textit{\enquote{Schnittstellen nach außen müssen geklärt sein.}} (siehe \combref{appendix:t-22})
}

Weiterführend wurde von Herr Troppmann eine Struktur hervorgehoben, die Abläufe genauer beschreibt.\footnote{
\textit{\enquote{[…] zumindest benötigt man Strukturen, sodass man weiß, was wie abläuft.}} (siehe \combref{appendix:t-22})
} Diesen Ansatz erwähnte auch Herr Rahles, als er die Notwendigkeit von Projektmanagement ansprach.\footnote{
\textit{\enquote{[…] es ist einfach sinnvoll zu gucken, dass […] die Teams entsprechend des Projektmanagements richtig aufgestellt sind.}} (siehe \combref{appendix:r-22})
}

\label{sec:eigenverantwortlich} \label{sec:schnittstelle}
Daraus lässt sich schlussfolgern:
\begin{itemize}
	\item  Strukturen sind notwendig, damit Teams separat arbeiten können.
	\item Schnittstellen müssen definiert und beschrieben sein.
	\item Teams sind zwar nie ganz unabhängig, arbeiten aber eigenverantwortlich.
\end{itemize}

