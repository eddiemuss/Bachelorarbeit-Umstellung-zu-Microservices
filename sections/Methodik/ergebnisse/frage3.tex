\subsubsection{Frage 3}

\textit{Gibt es in Ihren Augen irgendwelche technischen Anforderungen die PluraPolit erfüllen sollte?}

Als wohl wichtigste technische Anforderung stellt sich das Monitoring heraus. Demnach beschrieben alle drei Experten, die Notwendigkeit die Kommunikation der Services zu Monitoren (siehe \combref{appendix:t-32} und \combref{appendix:s-16}). Besonders Herr Rahles verwies auf diese Bedingung hin.\footnote{
\textit{\enquote{Monitoring ist das A und O. Das heißt mit Anstieg der Komplexität, muss ich sicher sein, dass ich auch diese Komplexität hinreichend überblicken kann.}} (siehe \combref{appendix:r-31})
}

Auch Sicherheit wurde von Herrn Rahles als Grundvoraussetzung gesehen. Dabei sah er es jedoch als Grundvoraussetzung für jegliche Infrastruktur und nicht nur explizit für Microservices.\footnote{
\textit{\enquote{Sicherheit gehört für mich zu den Grundvoraussetzungen, egal über welche Architektur wir reden.}} (siehe \combref{appendix:r-32})
}

Herr Schlaak und Herr Rahles hoben beide ein gut implementiertes Fehlermanagement hervor. So beschrieb Herr Schlaak und Herr Rahles, dass Services \textit{\enquote{vernünftige Fehlermeldungen}} zurück geben sollten (siehe \combref{appendix:s-18} und \combref{appendix:r-34}).

%Todo Schlussfolgerung der Robustheit von Schnittstellen hervorheben, sowie dass Services autak sein sollen.

Die technischen Anforderungen Monitoring, Sicherheit und Fehlermanagement, lassen schlussfolgern, dass Kenntnisse in der Implementierung und Managen vorausgesetzt ist. Eine Aussage, die von Herrn Troppman noch einmal bestätigt wurde.\footnote{
\textit{\enquote{[…] es [braucht] einen Entwickler, der sich mit dem Aufbau der Infrastruktur auskennt.}} (siehe \combref{appendix:t-33})
}

Herr Rahles verstärkt ebenfalls diese Schlussfolgerung, indem er die \textit{\enquote{heutzutage geltenden Best Practices als Anforderung}} sieht (siehe \combref{appendix:r-26}). Er führt diese Bedingung jedoch noch weiter aus und zählt automatische Tests, sowie ein automatischer Integration- und Deployment-Prozess als erforderlich, um nicht Zeit beim Aussetzten des Systems zu verliehren (siehe \combref{appendix:r-27}).

%Todo Folgerung über Technisches Wissen zum Aufbau und Implementierung einer solchen Infrastruktur
