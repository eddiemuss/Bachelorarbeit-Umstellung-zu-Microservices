\subsubsection{Frage 1}

\textit{Haben Sie das Gefühl, dass es Bedingungen gibt, die PluraPolit erfüllen sollte, bevor es ihre Softwarearchitektur zu einer Microservicearchitektur umstellt und wenn ja, welche Bedingungen empfinden Sie als wichtig?}

Alle drei Experten haben diese Frage mit Ja beantwortet und weiterführend Bedingungen genannt (siehe \combref{appendix:r-1}, \combref{appendix:troppman} und \combref{appendix:schlaak}).

Sowohl Sebastian Schlaak, als auch Alexander Troppman haben auf die Frage geantwortet, dass sich erst Microservices in betracht ziehen, wenn inhaltlich unterschiedliche Anwendungen vorliegen. So beschreibt Herr Schlaak in seinem Interview: \textit{\enquote{ich glaube, dass wäre eine Bedingung, wenn man sagt: […] ich habe etwas [eine neue Funktion], was […] ein ganz anderen Zweck erfüllt […].}} (siehe \combref{appendix:s-5}). Herr Troppman fasst es in seinem Interview wie folgt zusammen:  \textit{\enquote{Also wie gesagt, ich brauche eine technische Trennung […]}} und verwies auf die Trennung zwischen unterschiedlichen logischen Abläufen (siehe \combref{appendix:t-1}).

Als zweite Bedingung wurde die Wirtschaftlichkeit genannt. Sie wurde einmal von Christoph Rahles und Alexander Troppman hervorgehoben. Folglich führte Herr Rahles in seinem Interview an, dass es jemanden geben muss, \textit{\enquote{der […] sich die Frage stellt: Ist es wirtschaftlich, Ja oder Nein?}} (siehe \combref{appendix:r-3}). Herr Troppman hielt es wie folgt fest: \textit{\enquote{[…] ich muss ein Business Case haben, dass sich das auch lohnt.}} (siehe \combref{appendix:t-2}). Demnach lässt zu schlussfolgern, dass der wirtschaftliche Nutzen ein wesentliche Voraussetzung für den Einsatz von Microservices ist.

Allein Herr Rahles verwies bei dieser Frage darauf, dass die Entscheidung für Microservices vom “Reifegrad des Geschäftsmodells” abhängt (siehe \combref{appendix:r-1}).
