\subsection{Auswahl der Experten}

Für das Experteninterview wurden Christoph Rahles, Alexander Troppman und Sebastian Schlaak befragt. Es wurde sich für sie entschieden, da alle drei jahrelange Erfahrung im Bereich der Microservicearchitektur haben und seit vielen Jahren in der Start-up-Branche arbeiten. Sowohl Herr Rahles, Herr Troppman und Herr Schlaak sind Senior Software Developer mit Erfahrungen im Management. So haben alle drei mehre Jahre als Chief Technology Officer (CTO) gearbeitet und können fundierte Aussagen über Software architektonische Entscheidungen geben. Darüber hinaus hält Herr Troppman Vorträge in denen er erklärt, wie man mit Hilfe der Programmiersprache Golang leichtgewichtige Microservices erstellt kann.

Des Weiteren wurden alle drei ausgewählt, da sie PluraPolit kennen und bei der Entwicklung beteiligt waren. So hat Herr Rahles insbesondere in der Anfangsphase PluraPolit geholfen, die Softwarearchitektur mit aufzubauen und kennt diese detailliert.

Mit Herrn Trapper hat PluraPolit erst vor kurzen gemeinsam in einem Hackathon gearbeitet. Bei diesem wurde die Plattform konzeptionell weiter entwickelt, sodass sie auch für den Schulunterricht eingesetzt werden kann. Folglich kennt auch Herr Troppman den aktuellen Stand der Bildungsplattform.

Mit Herrn Schlaak stand PluraPolit gelegentlich im Austausch. Sowohl ich als auch Robin Zuschke haben vor der Zeit bei PluraPolit mit Herrn Schlaak zusammen gearbeitet und sind seitdem im freundschaftlichen Kontakt. Folglich kannte Herr Schlaak PluraPolit. 

Auch kannten sich Herr Schlaak und Herr Rahles vor dem Interview. Sie haben von 2011 bis 2013 gemeinsam bei Käuferportal gearbeitet und sind seitdem befreundet. Es wurde jedoch darauf geachtet, dass sie sich vor den Interviews nicht austauschen und ihre Antworten unabhängig gegeben wurden.

Ausgenommen der Verbindung zwischen Herrn Rahles und Herrn Schlaak, kannten sich die Experten nicht.