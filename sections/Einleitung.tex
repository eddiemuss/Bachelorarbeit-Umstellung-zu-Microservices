\section{Einleitung}

Für ein junges Unternehmen, im Bereich der digitalen Produktentwicklung, ist es wichtig, schnell Ideen umzusetzen und erstes Feedback zu erhalten. Dies helft bei der Bestätigung von Annahmen und bei der Beschaffung von Investoren.

Diese schnelle Softwareentwicklung führt jedoch zu einigen Abstrichen hinsichtlich der Qualität. So wird zu Beginn oftmals auf einen automatischen Prozess zur Bereitstellung der Applikation verzichtet und leicht umsetzbare Lösungen vor langfristigen bevorzugt. Auch hinsichtlich der Auswahl der Architektur wird anfängliche Performance als oberstes Auswahlkriterium bestimmt. Endet der Weg für ein Start-up nicht nach wenigen Monaten, dann müssen die ursprünglichen Entscheidungen hinterfragt werden.

Genau an diesem Punkt ist das junge Unternehmen PluraPolit, welches sich erst Mitte letzten Jahres gegründet hat und innerhalb von wenigen Monaten ein fertiges Produkt entwickelte. PluraPolit hat sich zur Aufgabe gestellt eine Bildungsplattform für Jung- und Erstwähler zu Entwicklern und bei der Meinungsbildung zu unterstützen. Gefördert wird das Projekt von der Zentrale für politische Bildung und ist politisch unabhängig. Des Weiteren handelt es sich bei PluraPolit um ein gemeinnütziges Unternehmen, welches keine Absicht verfolgt, Gewinne auszuzahlen. Ich bin seit Anfang Januar an diesem Projekt beteiligt und begleite es seitdem als Frontendentwickler. Gemeinsam mit einem der drei Gründer sind wir zu zweit die technische Abteilung des Unternehmens und kümmern uns um die Weiterentwicklung der Plattform.

Die Inhalte der Plattform werden von den zwei anderen Gründern gepflegt und eingeholt. Es handelt sich dabei um neun verschiedene Tonaufnahmen zu einer Frage. Die Audioaufzeichnung kommen ausschließlich von Politikern und beziehen sich auf eine aktuelle politische Fragestellung. So gibt es zum Beispiel das Thema: Sollte der öffentlich-rechtliche Rundfunk abgeschafft werden? Die jeweiligen Politikerinnen und Politiker werden direkt zu einem Statement angefragt, welches anschließend ohne inhaltliche Veränderung auf die Webseite geladen wird. Angefragt werden immer alle Parteien, die im Bundestag vertreten sind. Neben Fragen, die ausschließlich von Politikern diskutiert werden, kommen gleichermaßen Themen auf die Plattform, die von den jeweiligen Expertinnen und Experten beantwortet werden. So gibt es bei der oben genannten Fragestellung auch eine Äußerung des Vertreters der Landesrundfunkanstalten ARD. Im Gegensatz zu anderen Anbieter für Nachrichten rund um Politik, stellt PluraPolit ausschließlich Sprachnachrichten auf die Plattform, die von jeweiligen Expertinnen und Experten kommen. Es wurde sich exklusiv für das Medium Tonaufnahme entscheiden, um eine junge Zielgruppe anzusprechen und die einzelnen Beiträge wie einen Podcast hören zu können.

\subsection{Problemstellung}

Umgesetzt wurde die Plattform in Ruby on Rails im Backend und React.js\footnotemark im Frontend. Dabei liefert das Backend auf Anfrage Inhalte an das Frontend und kümmert sich um die Speicherung von Daten. Das Frontend im Gegensatz fordert beim Laden der Webseite alle genötigten Informationen an und stellt sie anschließend dar. Trotz dieser Einteilung handelt es sich um eine Applikation mit gemeinsamer Codebase und einem Bereitstellungsprozess.

\footnotetext{
React.js ist eine JavaScript Bibliothek zum Erstellen von Benutzeroberflächen. Diese verwaltet die Darstellung im HTML-DOM und ermöglicht dem Entwickler Informationen zwischen Funktionen zu administrieren \parencite{react_webpage}.
}

Gehostet werden die Applikationen über den Cloud-Computing-Anbieter Amazon Web Services\footnotemark (AWS).
\footnotetext{AWS ist ein Tochterunternehmen des Online-Versandhändlers Amazon mit einer Vielzahl an Diensten im Bereich Cloud-Computing \parencite{amazon_homepage}.}
	Es wurde sich für diesen Dienst entschieden, um möglichst geringe Fixkosten zu haben und bei belieben die Kapazität ändern zu können. Die Anwendung wird in einem Docker-Container\footnote{Docker-Container sind isolierte virtuelle Umgebungen, in der eine Anwendung separat vom System des Rechners betrieben wird. Dadurch können Applikationen leicht von einem Computer zu einem Hosting Dienst geladen werden \parencite{docker_container}.} gespeichert und per Github Actions\footnote{Github Actions ist ein Software Dienst von Github, welches hilft Prozesse zu automatisieren. Es kann zum Beispiel zum automatischen Bereitstellen einer Webseite verwendet werden \parencite{github_actions}.} an AWS geliefert. Dort wird die Applikation in das Elastic Container Service\footnote{Elastic Container Service ist ein Container-Orchestrierungs-Service von Amazon Web Services, mit dessen Hilfe Container skalierbar verwaltet werden können \parencite{aws_ecs}.} (ECS) geladen und von Fargate\footnote{Fargate ist eine Serverless-Datenverarbeitungs-Engine, welche Container im Rahmen der vordefinierten Parameter verwaltet. So werden zum Beispiel durch diesen Dienst bei erhöhtem Bedarf neue Instanzen bereitgestellt und bei Verlust von Last Container-Instanzen eliminiert \parencite{aws_fargate}.} verwaltet. Die Daten werden in einer PostgreSQL\footnote{PostgreSQL ist eine objektrelationale Datenbank, welche sowohl Elemente einer relationalen als auch einer Objektdatenbank besitzt \parencite{postgresql}.} Datenbank abgespeichert, die auf einer Relational Database Service\footnote{RDS ist ein Service von Amazon Web Services, mit dessen Hilfe relationale Datenbanken verwaltet werden. Der Dienst ermöglicht das Aufsetzen, Managen und Skalieren von Datenbanken, wie zum Beispiel MySQL, MariaDB und PostgreSQL \parencite[vlg.][S.161 f.]{baron_aws_2016}.} (RDS) Instanz hinterlegt ist. Bilder und Tonaufnahmen werden in einem Simple Storage Service-Bucket\footnote{Der Speicherdienst von AWS S3 ist einer der ersten Dienste des Cloud-Computing-Anbieters. Er erleichtert die Speicherung von Objekten in der Cloud jeglichen Formats und lässt sich einfach verwalten. Die verwendete Speichermenge ist dynamisch und richtet sich automatisch nach der Größe der Dateien \parencite[vlg.][S. 23]{baron_aws_2016}.} (S3) gespeichert und stehen der Webseite per URL zur Verfügung. Um automatisch zu jedem Beitrag Intros zu generieren, wurde eine AWS Lambda\footnote{Amazon Lambda ist ein Service von AWS, über welchen Funktionalität innerhalb der Cloud ausgeführt wird. Es kann sich dabei um ein Service der Serverless ist, was bedeutet, dass sich nicht um den Server gekümmert werden muss. Somit können kleine Programme mit wenig Aufwand ausgeführt werden \parencites[vlg.][Kap. 15.3]{wolff_microservices_2018}{aws_lambda}.} Funktion geschrieben, die aus der Basis von Beschriftungstexten für jede Audioaufzeichnung eine weitere Aufnahme für die Einleitung erstellt.

Mit wachsender Codebase erhöht sich der Aufwand, der notwendig ist, um neue Funktionen zu entwickeln und zu implementieren. Dies liegt besonders daran, dass sich im Laufe der Entwicklung viele Abhängigkeiten zwischen Klassen und Methoden hervorgetan haben. Hierdurch steigt der Aufwand, der nötig ist, um sich in den Quellcode einzuarbeiten. Verursacht wird diese Korrespektivität, indem im Frontend die Funktionen und Klassen in logisch getrennte Bausteine geteilt und an mehreren stellen verwendet werden. Dies ermöglicht zwar eine schnelle Entwicklung, führt jedoch dazu, dass eine Veränderung einer Komponente Änderungen an mehreren Stellen auslöst. Diese Abhängigkeiten machen es mit steigender Menge an Sourcecode, immer komplexer weitere Funktionen umzusetzen, ohne bestehende Logik zu verändern. Hinzukommt, dass neben dem eigenen Quellcode auch externe Funktionalitäten genutzt werden, welche durch den Paketverwaltungsdienst von Node.js \parencite{nodejs} npm installiert werden.

Diese werden jedoch nur in Teilen der Anwendung verwendet, werden allerdings zum gesamten Frontend hinzugefügt. Ingesamt verlangsamt es die Bereitstellung der Applikation, da sie während des Prozesses installiert werden müssen. Für eine schnelle Entwicklung ist es somit wichtig, einen rapiden Bereitstellungsprozess zu entwickeln und die Zahl der externen Pakete auf das Nötigste zu begrenzen.

Um in Zukunft eine schnelle Weiterentwicklung der Applikation sicherzustellen, hat PluraPolit beschlossen, den aktuellen Aufbau in eine Microservice-Architektur zu ändern und die gesamte Plattform in inhaltlich getrennte Module zu teilen.

\subsection{Zielsetzung}

Schon im Jahr 2005 hat Peter Rodgers auf der Web Services Edge Conference über Micro-Web Services referiert. Er kombinierte die Konzepte der Service-Orientierten-Architektur (SOA) mit den der Unix-Philosophie und sprach von verbundenen REST-Services. Er versprach sich dadurch eine Verbesserung der Flexibilität der Service-Orientierten-Architektur \parencite[vlg.][]{rodgers_peter}. Erstmalig 2011 wurde dieser Ansatz als Microservice-Architektur bezeichnet \parencite[vlg.][]{dragoni_microservices_2017}. Ab 2013 entwickelte sich rund um das Thema eine immer größer werdendes Interesse, welches dazu führte, dass mehr Blogposts, Bücher, sowie wissenschaftliche Arbeiten geschrieben wurden. Somit sind die Definition und die Charakteristiken bis ins Detail beleuchtet. Des Weiteren gibt es einige Beispiele von bekannten Unternehmen, wie Netflix und Amazon, die die Herausforderungen der Überführung ihres Systems zu einer Microservice-Architektur beschreiben.

Trotz der Informationslage ist jedoch noch relativ unbekannt, ob auch Start-ups Microservices umsetzen können und welche Bedingungen dafür erforderlich wären. Es gibt kaum Erfahrungen, die es PluraPolit ermöglicht abzuschätzen, ob sich eine Umstellung zum aktuellen Zeitpunkt lohnt und welche Eigenschaften ein Unternehmen erfüllen müsste.

Aus diesem Grund ist das Ziel dieser Arbeit für PluraPolit die Bedingungen zu ermitteln, die für eine mögliche Umstellung erforderlich wären und eine klare Bewertung für die Sinnhaftigkeit des Vorhabens abzugeben. Insbesondere das Ausarbeiten der notwendigen Anforderungen an ein Unternehmen, welches sein System von einer monolithen Architektur zu einer Microservices-Architektur umstellen möchte, soll PluraPolit und anderen Start-ups helfen bewerten zu können, ob sich eine solche Überführung lohnt.

\subsection{Vorgehen}

Die Arbeit teilt sich in drei Bereiche auf: Den theoretischen Rahmen, die Methodik und die Auswertung.  So wird im ersten Abschnitt die theoretische Grundlage für Microservices gebildet. Es werden einzelne wichtige Merkmale beleuchtet und beschrieben. Außerdem wird ein Vergleich zwischen der aktuellen Software-Architektur des Unternehmens und Microservices erstellt. Anschließend werden aus den Merkmalen und der Gegenüberstellung wichtige Bedingungen für die Umstellung zu einer Microservice-Architektur abgeleitet, welche Grundlage für die Einschätzung sind.

Im nächsten Abschnitt werden diese Bedingungen im Rahmen einer qualitativen Befragung von Experten im Bereich Microservices eingeschätzt und bewertet. Hierfür werden Interviews durchgeführt. Es wird beschrieben, welche Experten ausgewählt werden und welche Expertise sie mitbringen. Des Weiteren werden die einzelnen Interviewfragen vorgestellt und deren Zusammenhang zur Zielsetzung erklärt. Dadurch wird deutlich, welchen Einfluss die Expertenaussagen auf die Einschätzung für PluraPolit haben.

Abschließend werden die Aussagen aus den Befragungen mit der theoretischen Ausarbeitung verglichen und auf PluraPolit bezogen. Hierfür wird die Umsetzbarkeit für das junge Unternehmen hinterfragt und die Auswertung diskutiert. Neben Microservices wird auch SOA als alternative Lösung vorgestellt. Beendet wird die These mit einer Einschätzung für PluraPolit, in der eine klare Beurteilung für oder gegen eine Überführung abgegeben wird.