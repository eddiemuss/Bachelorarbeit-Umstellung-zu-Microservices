\section{Fazit}

Ziel der Bachelorarbeit

Methodik

Ergebnisse

Bezug zu PluraPolit

Empfehlung für PluraPolit

Microservices sind vielschichtiger als Monolithen. Daher müssen Punkt ... beachtet werden.
Microserves sind ideal um die Komplexität von vielseitigen Unternehmen zu beherrschen.
Demnach muss vor der Umstellung folgende Bedingungen erfüllt sein.

Diese Bedingungen treffen für PluraPolit nicht zu. Deswegen wird für PluraPolit eine Umstellung abgeraten.


Abschnitt ... zeigt, dass die Plattform von PluraPolit eine Monolith ist, welcher teilweise Modular aufgebaut ist und vereinzelnte externe Services von AWS nutzt.
Aus der Problemstellung geht hervor, dass ....
Experten bewerten die Komplexität der Plattform als zu gering, um eine Notwendigkeit zu sehen Microervices umzusetzen.
PluraPolit nutzt AWS, welches die Anforderung an Sicherheit und Infrastruktur erfüllt. Kenntnisse über den Aufbau einer solchen Infrastruktur sind jedoch nicht vorhanden.
Die Ziele des Unternehmen, zielen darauf ab weiterführende Funktionen zu etablieren. Zum aktuellen Zeitpunkt kann jedoch kein wirtschaftlicher Mehrwerkt proknostizierbar werden.

Der Mangel an Kenntniss über den Aufbau und das Verwalten einer Microservicearchitektur würde für PluraPolit bedeuten, dass entweder die Entwickler sich weiterbilden müssen oder externe Fachkräfte eingekauft werden. In beiden Fällen würde es für PluraPolit Kosten bedeuten ohne eine Komplexität des Geschäftsmodells vorleigt, die eine Aufteilung rechtwertigt.

Abschnitt ... zeigt, dass die Problematik des Koppling und Koheränz auch durch ein modularen Monolithen gelöst werden kann.

Die Experten empholen die Trennung zwischen Front- und Backend.
Demnach ein erster Schritt in zu einem verteilten System.