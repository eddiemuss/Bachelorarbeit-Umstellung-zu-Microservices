\subsection{Start-up}
\label{sec:start-up}

Wie in \cref{sec:ddd} aufgezeigt, sind Softwarearchitektur und Geschäftsprozesse ent mit einander verbunden. So soll an dieser Stelle der Arbeit zum Verständnis des Ausmaßes der Umstellung, die Situation eines Start-up beleuchtet werden, da \textit{\enquote{hart zu verändernde Entscheidungen}}\footnote{Direkter Bezug zur Definition von Martin Fowler aus \cref{sec:software-architect-definition}.} stets im Kontext der Unternehmenssituation getroffen werden sollten.

Nach dem Wirtschaftslexikon Gabler ist ein Start-up, oder auch Start-up-Unternehmen, ein junges, noch nicht etabliertes Unternehmen, welches eine innovative Geschäftsidee verwirklichen möchte \parencite[vgl.][]{achleitner_start-up_2018}. Ein Start-up agiert in einem jungen oder noch nicht existierenden Markt und muss erst ein funktionierendes Geschäftsmodell entwickeln. Wird ein solches gefunden und implementiert, gilt das Unternehmen nicht mehr als Start-up \parencite[vgl.][]{wiki_start-up-unternehmen_2020}.

Es ist jedoch schwierig, ein funktionierendes Geschäftsmodell zu finden und die meisten der Start-ups melden nach  wenigen Jahren Insolvenz an. Neun von zehn Unternehmen scheitern innerhalb der ersten drei Jahre \parencite[vgl.][]{patel_startups-fail_2015}.

Erfolgreiche Start-ups zeichnen sich durch adaptives Verhalten aus. Zwei Drittel der erfolgreichen Start-ups ändern  ihre ursprüngliche Geschäftsidee drastisch \parencite{mullins_getting_2009}. Daraus lässt sich ein signifikanter Zusammenhang, zwischen Erfolg und Flexibilität ableiten. Ein erfolgreiches Start-up arbeitet dynamisch und fokussiert und sammelt schnell neue Erkenntnisse über die Bedürfnisse von Kunden.

Das Ziel eines Start-ups ist, ein Produkt zu entwickeln, das einen lukrativen Absatzmarkt besitzt, sowie für den Kunden einen Mehrwert zu generieren, für den dieser auch bezahlt.

Um dies zu erreichen, werden drei separate Phasen durchläuft \parencite[vgl.][S. 8 f.]{maurya_running_2012}: 
\begin{enumerate}
	\item Problem-Lösung Fit
	\item Produkt-Market Fit
	\item Skalierung
\end{enumerate}

Zu Phase 1 (Problem-Lösung Fit):
Es wird ermittelt, ob ein lösungswertes Problem vorliegt. Hierfür werden qualitative Kundenbefragungen und Kundeninterviews durchgeführt \parencite[vgl.][S. 170 ff.]{croll_lean_2013}.

Zu Phase 2 (Produkt-Market Fit):
 Aus den Erkenntnissen der ersten Phase wird der minimale Funktionsumfang bestimmt und umgesetzt. Dieses sogenannte Minimum Viable Product (MVP) wird daraufhin für Anwendertests verwendet. Mit Hilfe der Ergebnisse aus den Tests wird abermals eine Anpassung des MVP genommen, das für weitere Kundentests eingesetzt wird. Dieser Prozess wird solange wiederholt, bis ein Produkt entwickelt wurde, welches das ursprüngliche Problem löst, oder Geld und Ressourcen ausgehen \parencite[vgl.][S. 28]{croll_lean_2013}.

In Phase 3 (Skalierung):
Das Start-up wird hinsichtlich Technologie, Struktur und Produkt angepasst, damit es in Bezug auf Kundenbedarf, Personal und Funktionsumfang ideal wachsen kann \parencite[vgl.][S. 9]{maurya_running_2012}.

Besondere die Phasen Eins und Zwei zeichnen sich durch schnelles, fokussiertes Lernen aus, welches in kontinuierlichen Anpassungen des Geschäftsmodells endet. Folglich kommt es in diesen beiden Phasen zu ständigen Änderungen des Produkts und Geschäftsprozesses. Erst nach dem Erreichen des Produkt-Market Fits stabilisiert sich das Produkt und der Fokus wird auf Skalierung gelegt.

Im Gegensatz zu den meisten Start-ups, ist PluraPolit ein gemeinnütziges Unternehmen. Der Fokus liegt deshalb nicht auf dem Erwirtschaften von Gewinn sondern auf der Maximierung der Kundeninteraktion. Da Interaktion sich aus der Akzeptanz der Anwender ergibt, ist PluraPolit bestrebt, ein Produkt-Market Fit zu erreichen. Zum aktuellen Zeitpunkt befindet es sich jedoch noch in Phase Zwei.
