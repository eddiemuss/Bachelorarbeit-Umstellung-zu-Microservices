\subsection{Microservices}

Nachdem nun die einzelne Architekturstile vorgestellt wurden und das System von PluraPolit eingeordnet wurde, wird nun Microservices vorgestellt.

Dabei handelt es sich um eine Architekturstil, der ein Vertreter der verteilten Systeme ist und sich historisch aus der Service Orientierten Architektur abgeleitet hat. (Siehe Zielstellung)

Eberhard Wolff beschreibt Microserivces als Ansatz Software in einzelne Module zu teilen und definiert es als Modularisierungskonzept, welches Einfluss auf die Unternehmesorganisation und Software-Entwicklungsprozess hat. Dabei ist jedes Module ein eigenes Programm.

Sam Newman schließt sich Wolff an und beschreibt Microservices als voneinander unabhängig einsetzbare Dienste, die um eine Geschäftsdomäne herum modelliert sind.

Somit beschreiben beide Microservices als ein System aus einzelnen unabhängigen Services, die sich an ein Geschäftsdomäne richten. Insbesondere die Abhängigkeit zum Geschätsprozess wird nachfolgend näher beschrieben.

% Todo Literaturnachweis einfügen

\subsubsection{Zusammenhang und Verknüpfung}

In diesem Abschnitt möchte ich diese beiden Begriffe erklären und noch einmal den Zusammenhang zwischen Funktionlität hervorheben. Es soll klar werden wie Microservices geteilt werden und was es bedeutet Logig hinter definierten Schnitstellen zu verstecken.

Abschließend zu diesem Abschnitt soll die dritte Bedingung für Microservices erstellt werden: \textbf{eine in teamsgeteilte, gemischte Unternehmensstruktur}

\subsubsection{Conway's Law}

Die Abhängigkeit der Kommunikationsstruktur im Unternehmen zur verwendeten Software Architektur.

\begin{itemize}
	\item Begriff erklären
	\item Auswirkungen für die Teamstruktur beschreiben
\end{itemize}

\subsubsection{Bounded Contexts}

An diesem Punkt sollte eine Definition des Begriffes Bounded Contexts gegeben werden. Dies ist insbesondere wichtig, da Domain Driven Design darauf aufbaut und Bounded Contexts umsetzt.

\subsubsection{Domain Driven Design}

\begin{itemize}
	\item Was ist DDD?
	\item Was sind Bounded Contexts?
	\item 	Bespiele dazu.
	\item Schlussfolgerung für PluraPolit
\end{itemize}
