\subsection{Microservices}

%	Todo Literatur finden und Abgrenzung schreiben. Es gab in dem Buch von Newman einen schnitt über die Geschichte und somit auch die Entwicklung des Begriffs.

%Todo der erste Satz muss noch einmal überprüft werden. Er kann so nicht richtig sein.
Der Begriff Microservices wird mehrdeutig verwendet. So wird je nach Perspektive entweder eine Softwarearchitektur oder eine Komponente einer solchen Architektur beschrieben. Eng verbunden mit diesem Begriff sind auch dynamische Systeme und Konzeptionierung von Unternehmensstrukturen. Demnach sieht Eberhard Wolf unter Microservices ein Modellierungskonzept, welches dazu dient größere Softwaresysteme in kleinere Einheiten zu teilen \parencite[vgl.][Kap. 1.1]{wolff_microservices_2018}. Dabei hat die Aufteilung Auswirkungen auf die Organisation als auch auf die Entwicklungsprozesse.

Zusammenfassend lässt sich sagen, dass der Begriff Microservices nicht einheitlich definiert ist und es sich zwei unterschiedliche Perspektiven ergeben. Zum einen beschreibt es ein Modellierungskonzept, welches Auswirkungen auf Unternehmenssturkur und Managemententscheidungen hat und zum anderen kennzeichnet es eine Software-Architektur, die in eigenständige Komponente geteilt ist.

In dieser Arbeit werden beiden Ansichten beleuchtet, um infolgedessen eine umfassende Einschätzung für PluraPolit zu geben.
