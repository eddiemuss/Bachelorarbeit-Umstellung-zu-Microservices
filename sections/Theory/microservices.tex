\subsection{Microservices}

Nachdem nun die einzelne Architekturstile vorgestellt wurden und das System von PluraPolit eingeordnet wurde, wird nun Microservices vorgestellt.

Dabei handelt es sich um eine Architekturstil, der ein Vertreter der verteilten Systeme ist und sich historisch aus der Service Orientierten Architektur abgeleitet hat (siehe \cref{sec:Zielsetzung}).

Eberhard Wolff beschreibt Microserivces als Ansatz Software in einzelne Module zu teilen und definiert es als Modularisierungskonzept, welches Einfluss auf die Unternehmesorganisation und Software-Entwicklungsprozess hat \parencite[vgl.][Kap. 1.1]{wolff_microservices_2018}. Dabei ist jedes Module ein eigenes Programm.

Sam Newman schließt sich Wolff an und beschreibt Microservices als voneinander unabhängig einsetzbare Dienste, die um eine Geschäftsdomäne herum modelliert sind \parencite[vgl.][Kap. 2.1]{newman_monolith_2019}.

Somit beschreiben beide Microservices als ein System aus einzelnen unabhängigen Services, die sich an ein Geschäftsdomäne richten. Insbesondere die Abhängigkeit zum Geschäftsprozess wird nachfolgend näher beschrieben.

\subsubsection{Zusammenhang und Verknüpfung}

Wenn es um das aufteilen von Software geht, ist es wichtig zu verstehen wie die einzelnen Funktionen und Klassen zusammenhängen und welche Verknüpfungen es zwischen ihnen gibt.

Dabei bezieht sich der Zusammenhang auf die funktionale Abhängigkeit zweier Funktionen oder Klassen \parencite[vgl.][Kap. 2.3.1]{newman_monolith_2019}. Somit liegt ein hoher Zusammenhang vor, wenn Quellcode anhand seiner logischen Zugehörigkeit geordnet ist. Eine konkrete Umsetzung dieses Bestreben ist im Abschnitt zum Model-View-Controller-Ansatz zu finden.

Verknüpfung hingegen beschreibt in welchem Maß, Funktionen und Klassen verbunden sind, ohne dass sie logisch zusammen gehören \parencite[vgl.][Kap. 2.3.2]{newman_monolith_2019}. Somit bezieht sich Verbindung ausschließlich auf eine technisch vorliegende Kopplung. Ein typisches Beispiel hier für ist, wenn ein Datenabruf an mehreren Stellen direkt über das Datenbankschema abläuft. Dadurch entsteht eine Abhängigkeit auf das Schema, sodass falls sich das Schema ändert auch die jeweiligen Aufrufe geändert werden müssen.

Insbesondere in monolithischen Systemen können viele Verknüpfungen entstehen, da keine festen Abgrenzungen zwischen einzelnen logischen Bereiche definiert sind. Somit kann es sein, dass ein Monolith, welches historisch wächst, keine klare Struktur aufzeigt. Es entsteht hieraus ein System, welches anfällig für Veränderungen ist.

Für ein Microservicesarchtektur ist jedoch das Bestreben klare Abgrenzung zu erlangen, und einzelne stabile Services zu etablieren. Diese sollen soweit es geht von einander entkoppelt funktionieren. Somit ist das Ziel für eine Microservicearchitektur einen hohen Zusammenhang bei geringer Verknüpfung zu besitzen. Dabei ist dies nicht nur die Zielsetzung für Microservices allgemein, sondern ein generelles Bestreben für stabile Systeme.\footnote{Dies bezieht sich auf das Gesetz von Constantine, welches besagt \textit{\enquote{A structure is stable if cohesion is strong and coupling is low.}} \parencite[S. 43]{endres_handbook_2003}.} Für Microservices bedeutet dies Konkret, dass ein Services ausschließlich aus funktional abhängigem Quellcode besteht und technische Implementierungen, wie zum Beispiel Datenbankstrukturen und Funktionsaufrufe, hinter klar definierten Schnittstellen versteckt sind.

\subsubsection{Conway's Law}

Die Abhängigkeit der Kommunikationsstruktur im Unternehmen zur verwendeten Software Architektur.

\begin{itemize}
	\item Begriff erklären
	\item Auswirkungen für die Teamstruktur beschreiben
\end{itemize}

\subsubsection{Bounded Contexts}

An diesem Punkt sollte eine Definition des Begriffes Bounded Contexts gegeben werden. Dies ist insbesondere wichtig, da Domain Driven Design darauf aufbaut und Bounded Contexts umsetzt.

\subsubsection{Domain Driven Design}

\begin{itemize}
	\item Was ist DDD?
	\item Was sind Bounded Contexts?
	\item 	Bespiele dazu.
	\item Schlussfolgerung für PluraPolit
\end{itemize}
