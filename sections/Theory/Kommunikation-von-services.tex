\subsection{Kommunikation von Services}
\label{sec:kommunikation}

In diesem Abschnitt möchte ich darauf eingehen welche anforderungen eine entkoppelte Infrastuktur an das Netztwerk und an die Kommunikation über standards hat.

Ich möchte die unterschiedlichen Standards vorstellen und die zweite Anforderung für Microservices hinsichtlich des Netzwerks stellen.

Es sollen auf folgende Kommunikationsmethoden eingegangen werden.
\begin{itemize}
	\item Request / Response
	\item REST
	\item HTTP-Operationen (GET, PUT, PATCH, POST, DELETE)
	\item gRPC (habe selbst keine Ahnung was das ist)
\end{itemize}

\subsubsection{CAP - Theorem}
\label{sec:cap}

System, Konsistenz, Verfügbarkeit und Partitionstoleranz können in einem verteilten System nicht gleichzeitig erfüllt sein.
Die einzelnen Begriffe erklären \begin{itemize}
	\item Konsistenz
	\item Verfügbarkeit
	\item Partitionstoleranz
\end{itemize}
