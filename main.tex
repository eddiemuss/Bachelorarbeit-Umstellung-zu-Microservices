\documentclass[a4paper, 12pt, headsepline]{scrartcl}

\usepackage[a4paper, left=3cm, right=2cm, top=2.5cm, bottom=2.5cm]{geometry}
%encoding
\usepackage[utf8]{inputenc}
\usepackage[T1]{fontenc}

%German-specific commands
\usepackage[ngerman]{babel}

%Hyphenation rules
\usepackage{hyphenat}

\usepackage{setspace}
\usepackage{parskip}

%Für quots
\usepackage[autostyle=true, german=quotes]{csquotes}

%Für Abbildungen
\usepackage{graphicx}

\usepackage{ragged2e}
\usepackage[format=plain,
      justification=RaggedRight,
      singlelinecheck=false]
     {caption}
  
% Inhaltsverwaltung wird eingefügt
\usepackage[style=authortitle, citestyle=authoryear]{biblatex}

%Klickbares Inhaltsverzeichnis
\usepackage[hidelinks]{hyperref}
\hypersetup{
	allcolors=black,
	linktoc=all,
}

%Um auf ein Bild innerhalb eines Textes referieren zu können
\usepackage[ngerman]{cleveref}

\addbibresource{references.bib}

\onehalfspacing

\begin{document}

\begin{titlepage}
\begin{center}

\textbf{\huge
Bachelor Thesis
}
\vspace{1.3cm}
		
zur Erlangung des akademischen Grades \\
Bachelor

\vspace{2cm}

\textbf{ \large
Technische Hochschule Wildau \\
Fachbereich Wirtschaft, Informatik, Recht \\
Studiengang Wirtschaftsinformatik (B. Sc.)
}

\vspace{1.4cm}

\textbf{Thema (Deutsch)} \\
Die Umstellung einer monolithischen in eine Microservices Architektur \\
am Beispiel von PluraPolit

\vspace{.5cm}

\textbf{Thema (Englisch)} \\
The conversion of a monolithic to a microservices architecture using the example of PluraPolit.

\vspace{1.5cm}

		\end{center}
\begin{description}
	\item [Autor] Edgar Muss
	\item [Matrikelnummer] 50033021
	\item [Seminargruppe] I1/16
	\item [Betreuer] Prof. Dr. Christian Müller
	\item [Zweitgutachter] Prof. Dr. Mike Steglich
	\item [Eingereicht am] 14.07.2020
\end{description}
\end{titlepage}

\newpage
\pagenumbering{Roman}

\section*{Abstract}

\newpage

\tableofcontents

\newpage
\pagenumbering{arabic}
\setcounter{page}{1}
\section{Einleitung}

Für ein junges Unternehmen im Bereich der digitalen Produktentwicklung ist es wichtig, Ideen schnell umzusetzen und ein erstes Feedback zu erhalten. Dies hilft bei der Bestätigung von Annahmen und bei der Gewinnung von Investoren.

Diese schnelle Softwareentwicklung führt jedoch zu einigen Abstrichen hinsichtlich der Qualität. So wird zu Beginn oftmals auf einen automatischen Prozess zur Bereitstellung der Applikation verzichtet und leicht umsetzbare Lösungen werden vor langfristigen bevorzugt. Auch hinsichtlich der Auswahl der Architektur wird anfängliche Performance als oberstes Auswahlkriterium bestimmt. Endet der Weg für ein Start-up nicht nach wenigen Monaten, dann müssen die ursprünglichen Entscheidungen hinterfragt werden.

Genau an diesem Punkt befindet sich das junge Unternehmen PluraPolit, das Mitte letzten Jahres gegründet wurde und innerhalb von wenigen Monaten ein fertiges Produkt entwickelte. PluraPolit hat sich zur Aufgabe gestellt, eine Bildungsplattform für Jung- und Erstwähler zu entwickeln und sie bei der Meinungsbildung zu unterstützen. Gefördert wird das Projekt von der Zentrale für politische Bildung und ist politisch unabhängig. Als gemeinnütziges Unternehmen verfolgt PluraPolit nicht die Absicht, Gewinne zu erzielen.

Ich bin seit Anfang Januar an diesem Projekt beteiligt und begleite es als Frontend-Entwickler. Gemeinsam mit einem der drei Gründer (Robin Zuschke) bilden wir zu zweit die technische Abteilung des Unternehmens und sind für die Weiterentwicklung der Plattform zuständig. Die Inhalte der Plattform werden von den zwei anderen Gründern eingepflegt. Im Mittelpunkt stehen aktuelle politische Fragestellungen wie z. B.: \textit{\enquote{Sollte der öffentlich-rechtliche Rundfunk abgeschafft werden?}} Hierzu werden Statements von Politikerinnen und Politikern aller Parteien, die im Bundestag vertreten sind, eingeholt und als Audioaufzeichnungen ohne inhaltliche Veränderung auf die Webseite geladen. Neben Fragen, die ausschließlich von Politikern diskutiert werden, kommen auch Themen auf die Plattform, die von den dafür ausgewiesenen Expertinnen und Experten beantwortet werden. So gibt es z. B. bei der oben genannten Fragestellung auch eine Stellungnahme des Vertreters der Landesrundfunkanstalten ARD. 

Im Gegensatz zu anderen Anbietern von Nachrichten rund um Politik, stellt PluraPolit ausschließlich Sprachnachrichten auf die Plattform. Die Entscheidung für das Medium Tonaufnahme wurde bewusst getroffen, um eine junge Zielgruppe anzusprechen und die einzelnen Beiträge wie einen Podcast hören zu können.

\subsection{Problemstellung}

Umgesetzt wurde die Plattform in Ruby on Rails\footnote{Weitere Informationen zu Ruby on Rails ist im \cref{sec:einordnung} beschrieben.} im Backend und React.js\footnotemark im Frontend. Dabei liefert das Backend auf Anfrage Inhalte an das Frontend und kümmert sich um die Speicherung von Daten. Das Frontend im Gegensatz fordert beim Laden der Webseite alle genötigten Informationen an und stellt sie anschließend dar. Trotz dieser Einteilung handelt es sich um eine Applikation mit gemeinsamer Codebase und einem Bereitstellungsprozess.

\footnotetext{
React.js ist eine JavaScript Bibliothek zum Erstellen von Benutzeroberflächen. Diese verwaltet die Darstellung im HTML-DOM und ermöglicht dem Entwickler, Informationen zwischen Funktionen zu administrieren \parencite{react_webpage}.
}

Gehostet werden die Applikationen über den Cloud-Computing-Anbieter Amazon Web Services\footnotemark (AWS).
\footnotetext{AWS ist ein Tochterunternehmen des Online-Versandhändlers Amazon mit einer Vielzahl an Diensten im Bereich Cloud-Computing \parencite{amazon_homepage}.}

Es wurde sich für diesen Dienst entschieden, um möglichst geringe Fixkosten zu haben und bei Bedarf die Kapazität ändern zu können. Die Anwendung wird in einem Docker-Container\footnote{Docker-Container sind isolierte virtuelle Umgebungen, in der eine Anwendung separat vom System des Rechners betrieben wird. Dadurch können Applikationen leicht von einem Computer zu einem Hosting Dienst geladen werden \parencite{docker_container}.} gespeichert und per Github Actions\footnote{Github Actions ist ein Software Dienst von Github, welches hilft Prozesse zu automatisieren. Es kann z. B. zum automatischen Bereitstellen einer Webseite verwendet werden \parencite{github_actions}.} an AWS geliefert. Dort wird die Applikation in den Elastic Container Service\footnote{Elastic Container Service ist ein Container-Orchestrierungs-Service von Amazon Web Services, mit dessen Hilfe Container skalierbar verwaltet werden können \parencite{aws_ecs}.} (ECS) geladen und von Fargate\footnote{Fargate ist eine Serverless-Datenverarbeitungs-Engine, welche Container im Rahmen der vordefinierten Parameter verwaltet. So werden z. B. durch diesen Dienst bei erhöhtem Bedarf neue Instanzen bereitgestellt und bei Verlust von Last Container-Instanzen eliminiert \parencite{aws_fargate}.} verwaltet. 
Die Daten werden in einer PostgreSQL\footnote{PostgreSQL ist eine objektrelationale Datenbank, welche sowohl Elemente einer relationalen als auch einer Objektdatenbank besitzt \parencite{postgresql}.} Datenbank abgespeichert, die auf einer Relational Database Service\footnote{RDS ist ein Service von Amazon Web Services, mit dessen Hilfe relationale Datenbanken verwaltet werden. Der Dienst ermöglicht das Aufsetzen, Managen und Skalieren von Datenbanken, wie z. B. MySQL, MariaDB und PostgreSQL \parencite[vgl.][S.161 f.]{baron_aws_2016}.} (RDS) Instanz hinterlegt ist. 
Bilder und Tonaufnahmen werden in einem Simple Storage Service Bucket\footnote{Der Speicherdienst von AWS S3 ist einer der ersten Dienste des Cloud-Computing-Anbieters. Er erleichtert die Speicherung von Objekten in der Cloud jeglichen Formats und lässt sich einfach verwalten. Die verwendete Speichermenge ist dynamisch und richtet sich automatisch nach der Größe der Dateien \parencite[vgl.][S. 23]{baron_aws_2016}.} (S3) gespeichert und stehen der Webseite per URL zur Verfügung. 
Um automatisch zu jedem Statement ein Intro zu generieren, wurde eine AWS Lambda\footnote{Amazon Lambda ist ein Service von AWS, über den Funktionalität innerhalb der Cloud ausgeführt wird. Es handelt sich dabei um ein Service der Serverless ist, was bedeutet, dass man sich nicht um das Betriebssystem des Servers kümmern muss. Somit können kleine Programme mit wenig Aufwand ausgeführt werden \parencites[vgl.][Kap. 15.3]{wolff_microservices_2018}{aws_lambda}.} Funktion geschrieben, die auf der Basis von Beschriftungstexten für jede Audioaufzeichnung eine weitere Aufnahme für die Einleitung erstellt.

Mit wachsender Codebase erhöht sich der Aufwand, der notwendig ist, um neue Funktionen zu entwickeln und zu implementieren. Dies liegt besonders daran, dass sich im Laufe der Entwicklung viele Abhängigkeiten zwischen Klassen und Methoden ergeben haben. Hierdurch steigt der Aufwand, der nötig ist, um sich in den Quellcode einzuarbeiten. Verursacht wird diese Abhängigkeit, indem im Frontend die Funktionen und Klassen in logisch getrennte Bausteine geteilt und an mehreren Stellen verwendet werden. Dies ermöglicht zwar eine schnelle Entwicklung, führt jedoch dazu, dass die Veränderung einer Komponente Änderungen an mehreren Stellen auslöst. Diese Relativitäten machen es mit steigender Menge an Sourcecode immer komplexer, weitere Funktionen umzusetzen ohne die bestehende Logik zu verändern. Hinzukommt, dass neben dem eigenen Quellcode auch externe Funktionalitäten genutzt werden, welche durch den Paketverwaltungsdienst von Node.js \parencite{nodejs} npm installiert werden.

Diese werden jedoch nur in Teilen der Anwendung verwendet, werden allerdings zum gesamten Frontend hinzugefügt. Insgesamt verlangsamt es die Bereitstellung der Applikation, da sie während des Prozesses installiert werden müssen. Für eine schnelle Entwicklung ist es somit wichtig die Zahl der externen Pakete auf das Nötigste zu begrenzen.

Um in Zukunft eine schnelle Weiterentwicklung der Applikation sicherzustellen, hat PluraPolit beschlossen, den aktuellen Aufbau in eine Microservice-Architektur zu ändern und die gesamte Plattform in inhaltlich getrennte Module zu teilen.

\subsection{Zielsetzung}
\label{sec:zielsetzung}

Schon im Jahr 2005 hat Peter Rodgers auf der Web Services Edge Conference über Micro-Web Services referiert. Er kombinierte die Konzepte der Service-Orientierten-Architektur (SOA) mit denen der Unix-Philosophie und sprach von verbundenen REST-Services. Dadurch versprach er sich eine Verbesserung der Flexibilität der Service-Orientierten-Architektur \parencite[vgl.][]{rodgers_peter}. Erstmalig 2011 wurde dieser Ansatz als Microservice-Architektur bezeichnet \parencite[vgl.][]{dragoni_microservices_2017}. Ab 2013 entwickelte sich rund um das Thema ein immer größer werdendes Interesse, welches dazu führte, dass mehr Blogposts, Bücher, sowie wissenschaftliche Arbeiten geschrieben wurden. Somit sind die Definition und die Charakteristiken bis ins Detail beleuchtet. Des Weiteren gibt es einige Beispiele von bekannten Unternehmen, wie Netflix und Amazon, die die Herausforderungen der Überführung ihres Systems zu einer Microservice-Architektur beschreiben.

Trotz der Informationslage ist es noch unbekannt, ob auch Start-ups Microservices umsetzen können und welche Bedingungen dafür erforderlich wären. Es gibt kaum Erfahrungen, die es PluraPolit ermöglichen abzuschätzen, ob sich eine Umstellung zum aktuellen Zeitpunkt lohnt und welche Eigenschaften ein Unternehmen erfüllen muss.

Das Ziel dieser Arbeit ist es deshalb, für PluraPolit die Bedingungen zu ermitteln, die für eine mögliche Umstellung erforderlich sind und eine klare Bewertung für den Nutzen des Vorhabens abzugeben. Es sollen die notwendigen Anforderungen an ein Unternehmen ausgearbeitet werden, das sein System von einer monolithischen Architektur zu einer Microservice-Architektur umstellen möchte.


\subsection{Vorgehen}

Die Arbeit teilt sich in drei Bereiche auf: 
\begin{itemize}
	\item den theoretischen Rahmen,
	\item die Methodik und
	\item die Auswertung.
\end{itemize}

Im ersten Abschnitt wird die theoretische Grundlage für Microservices dargestellt. Es werden einzelne wichtige Merkmale beleuchtet und beschrieben. Aus den Merkmalen werden wichtige Bedingungen für die Umstellung zu einer Microservice-Architektur abgeleitet, welche Grundlage für die Experteninterviews sind.

Im nächsten Abschnitt werden diese Bedingungen im Rahmen einer qualitativen Befragung von Experten im Bereich Microservices eingeschätzt und bewertet. Hierfür werden Interviews durchgeführt. Es wird beschrieben, welche Experten ausgewählt werden und welche Expertise sie mitbringen. Die einzelnen Interviewfragen werden vorgestellt und deren Zusammenhang zur Zielsetzung erklärt. Dadurch wird deutlich, welchen Einfluss die Expertenaussagen auf die Einschätzung für PluraPolit haben.

Abschließend werden die Aussagen aus den Befragungen mit der theoretischen Ausarbeitung verglichen und auf PluraPolit bezogen. Beendet wird die Arbeit mit einer Einschätzung für PluraPolit, in der eine klare Beurteilung für oder gegen eine Umstellung abgegeben wird.


\newpage

\section{Theoretischer Rahmen}

Der Abschnitt umfasst ca. 16 Seiten.

\subsection{Software Architektur}

Seit dem die ersten Großrechner gebaut wurden und ein Projekt nicht mehr von einem Team allein entwickelt werden konnte, entstand der Bedarf komplexe Systeme aufzuteilen und zu strukturieren. So war es schon in den 60er Jahren notwendig, die Entwicklung des  Betriebssystems OS/360 von IBM in mehrere Team aufzuteilen und klare Schnittstellen zwischen den Teilen zu bestimmen \parencite{brooks_mythical_1995}. Es entwickelte sich daraus eine der ersten Anwendungen und Umsetzungen von  Softwarearchitektur, welche erstmalig 1969 bei einer Softwaretechnik Konferenz in Rom auch als solche bezeichnet wurde \parencite[vgl.][S. 12]{buxton_software_1970}. In den darauf folgenden Jahren wuchs das Interesse an der Thematik und die Anwendungen der Teilung und Strukturierung von Softwaresystemen.
Hieraus entstand die im Jahr 2000 veröffentlichte Norm IEEE1471:2000, welche am 15. Juli 2007 als  ISO/IEC  42010 übertragen wurde. In diesem Standard werden Anforderungen an die Beschreibung von System-, Software- und Unternehmensarchitekturen definiert \parencite{hilliard_isoiecieee_nodate}.

\subsubsection{Definition}

Helmut Balzert, einer der führenden Pioniere im Bereich Softwarearchitektur und Autor der Bücherreihe \textit{Lehrbuch der Softwaretechnik}, beschreibt diese als \textit{\enquote{eine strukturierte oder hierarchische Anordnung der Systemkomponenten, sowie Beschreibung ihrer Beziehungen}} \parencite[][S. 580]{balzert_lehrbuch_2011}. Ihm nach lässt sich somit jedes System in mehrere einzelne Komponenten teilen, welche untereinander in Verbindung stehen und gemeinsam das Gesamtsystem formen.

Paul Clements, Autor der Bücher \textit{Software Architecture in Practice und Documenting Software Architectures: Views and Beyond}, schließt sich Balzert an und beschreibt Softwarearchitektur als \textit{\enquote{Strukturen eines Softwaresystems: Softwareteile, die Beziehungen zwischen diesen und die Eigenschaften der Softwareteile und ihrer Beziehungen}} \parencite[][S. 23]{clements_documenting_2010}.

Somit definieren beide Softwarearchitektur als Strukturierung von einzelnen Komponenten, die untereinander in Beziehung stehen. Dabei können sowohl die Komponenten als auch die Beziehungen Eigenschaften besitzen. Die einzelnen Komponenten zusammen ergeben das Gesamtsystem, welches in einer bestimmten Struktur vorliegt und beschrieben wird. Folglich beinhaltet die Softwarearchitektur alle nötigen Informationen über die Struktur der einzelnen Systemkomponenten und deren Kommunikationen untereinander.

Wird ein Softwaresystem in Komponenten geteilt, welches jedoch selbst eine Funktionalität besitzt, bedeutet dies, dass auch diese Logik in Teile geteilt wird und jede einzelne Komponente einen Teil erfüllt. Um gleichermaßen den selben Funktionsumfang, wie das Gesamtsystem bewältigen zu können, müssen die einzelnen Komponenten zusammenarbeiten. Demnach ist es wichtig, die Zuständigkeit jeder einzelnen Komponente genau zu klären und die Abhängigkeiten zu bestimmen. Auf Grund der Abhängigkeiten werden im Anschluss Schnittstellen definiert, über welche die einzelnen Komponenten Informationen austauschen können.

Bei der Teilung in einzelne Komponenten können bekannte Architekturstile helfen, da sich diese im Laufe der Zeit entwickelt haben und gut dokumentiert sind. Architekturstile sich dabei einige Regeln zur Strukturierung eines Systems und fassen Merkmale eines IT-Systems, sowohl hinsichtlich der Komponenten, als auch ihrer Kommunikation zusammen \parencite[vlg.][S. 102]{starke_effektive_2015}. Seit Beginn der Softwarearchitektur hat sich eine Vielzahl an solchen Stilen entwickelt, die aus unterschiedlichen Intention entstanden sind. Jedes System hat dabei unterschiedliche Herangehensweisen, sowie Vor- als auch Nachteile. Da die Lösung eines Problem von der eigentlichem Umsetzung unabhängig ist, können auch unterschiedliche Architekturstile zur ein und der selben Lösung verwendet werden. Hinsichtlich der Auswahl des Stiles gibt es demnach keine richtige oder falsche Antwort. Es ist viel mehr das Abwiegen von Pro und Kontra, sowie eine persönliche Präferenz der Entwickler.

Da die unterschiedlichen Architekturstile nicht im  Fokus dieser Bachelor Thesis sind, werden nur folgende Stile betrachtet:

\begin{enumerate}
	\item Verteilte Systeme,
	\item Interaktionsorientierte Systeme,
	\item REST-Architektur,
	\item Monolithe Architektur
\end{enumerate}

\subsubsection{Verteilte Systeme}

Nach Andrew Tanenbaum werden verteilte Systeme als eine Menge unabhängiger Computer bezeichnet, die dem Benutzer wie ein einzelnes, kohärentes System erscheinen \parencite{tanenbaum_verteilte_2007}. Weiterführend beschreibt Gernot Starke die einzelnen Komponenten als entweder Verarbeitungs-, oder Speicherbausteine, die über definierte Schnittstellen innerhalb eines Kommunikationsnetz zusammenarbeiten \parencite[vlg.][S. 116]{starke_effektive_2015}. Im Gegensatz zur Definition von Softwarearchitektur grenzt sich das verteile System dadurch ab, dass von unabhängigen Computer geredet wird. Dadurch entstehen einige Vorteile. So lassen sich auf Grund der Unabhängigkeit, diese einzelnen Rechner unterschiedlich skalieren. Es entsteht dadurch ein Netzwerk an heterogenen Computern. Somit kann je nach Anforderung der einzelnen Komponenten, eine entsprechende Rechenleistung genutzt werden. Des Weiteren gibt es auf Grund der Verteilung eine gewisse Ausfallsicherheit. Diese entsteht, da das Gesamtsystem nicht durch eine, sondern durch mehrere Maschinen getragen wird. Dadurch können einzelne Computer ausfallen ohne, dass das Anwendungssystem ausfällt. Jedoch kann dadurch ein Teil der gesamten Funktionalität wegfallen, der ggf. für das System notwendig ist. Um dies zu verhindern, sollten geeignete Maßnahmen getroffen werden, indem zum Beispiel nicht allein ein Computer für die Funktionalität verantwortlich ist, sondern mehrere.

Jedoch entsteht mit der Verteilung auch ein Anstieg der Komplexität, sowohl bei dem Konzipieren des Systems, als auch bei der Wartung und Managen dessen. Außerdem muss nicht nur ein Rechner abgesichert werden, sondern ein ganzes Netzwerk. Dadurch entsteht ein höherer Aufwand zur Absicherung.

Die einzelnen Computer können über verschiedene Mechaniken miteinander kommunizieren: Einerseits durch direkten Aufruf entfernter Funktionalität und andererseits durch indirekten Austausch von Informationen \parencite[vlg.][S. 116]{starke_effektive_2015}. Dabei kann der Transfer synchron oder asynchron ablaufen. Bei einem synchronen Aufruf, der nur direkt ausgelöst wird, führt ein Computer über das Netzwerk die Funktionalität eines anderen aus und wartet auf dessen Antwort \parencite{synchrone_2018}.
Bei einem asynchronen Aufruf wird entweder direkt oder indirekt Logik eines anderen Computer aufgerufen, während der aufrufende Rechner, ohne auf die Antwort zu warten, weiter verarbeitet. Ist die aufgerufene Rechner fertig gibt er das Resultat zurück, welches vom ersten Computer aufgenommen und verarbeitet wird \parencite{wiki_asynchrone_2019}.

%Todo Mit Müller klären ob Wikipedia als Quelle gilt.

Der Austausch von Informationen und das Aufrufen von externer Funktionalität kommt in einem System dauerhaft vor. Dadurch besteht das Risiko, dass einzelne Informationen verloren gehen können und das System sicherstellen muss, dass das Anwendungssystem nicht darunter leidet.

\subsubsection{Interaktionsorientierte Systeme}

Interaktionsorientierte Systeme zeichnen sich dadurch aus, dass sie den Fokus auf die Interaktion zwischen Mensch und Maschine legen \parencite[vlg.][S. 124]{starke_effektive_2015}.
Ein viel verwendeter Vertreter hiervon ist der Model-View-Controller Ansatz. Hierbei werden die einzelnen Komponenten in drei unterschiedliche Kategorien eingeteilt, von dem jeweils ein Repräsentant vorhanden sein muss. Eingeteilt wird in die drei Kategorien: Model, View und Controller, unterdessen jede Gattung eine eigene Funktion besitzt. So kümmert sich das Model um die Datenspeicherung, den Datenabruf und Verarbeitung von Informationen. Davon getrennt sind die graphischen Darstellungen, welche durch Views definiert werden. Sie erhalten ihre Informationen vom Model. Das Verwalten der Benutzereingabe, sowie das weiterleiten zwischen einzelnen Views geschieht über den Controller. Er sorgt dafür, dass die Events oder Aktionen, die vom Benutzer über die Views auslöst werden, verarbeitet werden und führt entsprechende Datenverarbeitungen im Model aus. Abschließend updated er die erforderliche Darstellung.

Beim Model-View-Controller Ansatz handelt es sich um ein Muster, welches oft in der Softwarearchitektur verwendet wird. Im Unterschied zu verteilten Systemen stellt das Muster jedoch keine Anforderungen bezügliche der Hardware. Viel mehr beschreibt es eine Art den Quellcode hinsichtlich seiner Funktion zu teilen. Demnach kann dieser Stil auch für Codeteilung innerhalb einer Komponente verwendet werden und beschreibt nicht zwingend ein Architekturstil, der sich auf das Gesamtsystem besieht.

%Todo Einordnung und zusammenfassende Sätze hinzufügen
%Todo Warum sollte ich MVC verwenden, muss mehr hervorgehoben werden
%Todo Bilder einfügen

\subsubsection{REST-Architektur}

Neben den bislang genannten Architekturstilen gibt es eine Vielzahl von weiteren Strukturierungen, die sich erst in den letzten 20 Jahre entwickelt haben. Einer dieser Architekturstile ist die REST-Architektur, welche vom Miterfinder des HTTP-Standards Roy Fielding definiert wurde \parencite[][S. 128]{starke_effektive_2015}. Er beschrieb diesen Stil in seiner Dissertation an der Universität von Kalifornien im Jahr 2000 und charakterisiert ihn als Architekturstil fürs Web.

Dabei steht REST für \textit{Representationl State Transfer}, welches ein Architekturstil für verteilte Systeme beschreibt und auf der Server-Client Architektur aufbaut \parencite[][S. 76]{fielding_architectural_2000}. Server-Client Architektur beschreibt eine Ausprägung eines verteilten Systems, bei dem die Anwendung in Server und Clients geteilt werden \parencite[][S. 117]{starke_effektive_2015}.\footnote{Hier kommt ein Text zur Einteilung des Begriffes Server-Client Architektur.}

%Todo Fußnote zu Server-Client Architektur und damit Abgrenzung zum ursprünglichen Begriff des Mainfraims
%Todo Soll die Erklärung vorher kommen, oder hier als Einschub bleiben. Bei Müller nachfragen.

Ein Server ist dabei eine Komponente im Netzwerk, welches Services anbietet. Ein Service könnte zuständig sein, alle Informationen der hinterlegten Kunden auszugeben. Der Client hingegen konsumiert lediglich diese Informationen und dient dem Benutzer als Bedienungsoberfläche. Dies bedeutet, dass der Server nur passiv auf Anfragen vom Client wartet, während der Client selbst keine Informationen verarbeitet, sondern ausschließlich anzeigt.

Die REST-Architektur verwendet diese Aufteilung, um eine feste Trennung der Zuständigkeit zu integrieren \parencite[vgl.][S. 78]{fielding_architectural_2000}.

Die zweite Bedingung, die Fielding an den Architekturstil gestellt hat, ist dass die Kommunikation zustandslos, zu englisch (stateless), abläuft \parencite[][S. 78]{fielding_architectural_2000}. Dies bedeutet, dass die Nachrichten, die zwischen Server und Client ausgetauscht werden, alle nötigen Informationen beinhalten \parencite[][S. 128]{starke_effektive_2015}. Somit gibt der Server auf Anfrage des Clients stets die gleiche Antwort zurück, egal ob dieser zum ersten, oder wiederholten Mal angefragt wurde. Des Weiteren hängt die Antwort nicht vom Client ab.
Diese Entkopplung zwischen den Komponenten ermöglicht, dass die Aufgabe des Server, sowie des Clients durch mehrere Computer verrichtet werden kann und somit das System skalierbar ist \parencite[][S. 79]{fielding_architectural_2000}.

Der Hauptunterschied zwischen der REST-Architektur und anderen Stilen liegt jedoch in der genauen Bestimmung der zu verwendeten Kommunikationsschnittstellen. So bestimmt die REST-Architektur sehr explizit, welche From zur Kommunikation verwendet werden darf. Anders als andere Stile beruht der Aufruf von Methodiken nicht auf individuelle Funktionalität, sondern auf dem HTTP-Standard. Konkret bedeutet dies, dass die einzelnen Dienste des Servers sich an die HTTP-Optionen (GET, PUT, POST und DELETE) richten und keinen eigenen verwenden \parencite[vlg.][S. 128]{starke_effektive_2015}. Somit baut die REST-Architektur auf ein Kommunikationsstandard auf, der sich im Internet etabliert hat.

%Todo ggf. Mapping von CRUD mit einbauen

Auf Grundlade der standardisierten Kommunikation können zwischen Server und Client intelligente Zwischenstationen geschalten werden, die dazu zuständig sind häufig vorkommende Anfragen abzuspeichern \parencites[vlg.][S. 79 f.]{fielding_architectural_2000}[][S. 128]{starke_effektive_2015}. Somit lässt sich eine Vielzahl von Serveranfragen im vornherein beantworten.

Die Antwort des Servers erfolgt durch Repräsentationen der Daten, wovon es für jede Ressource mehr Formate gibt. So kann eine Schnittstelle abhängig des angeforderten Mediums, sowohl JSON­, als auch XML­ oder HTML zurück geben \parencite[vgl.][S. 128]{starke_effektive_2015}.

Verwendet wird die REST-Architektur ausschließlich für Anwendungen im Internet, da es auf die Anwendung des Hypertext Transfer Protokoll\footnote{Weitere Informationen zum Hypertext Transfer Protokoll kann unter folgender Literatur gefunden werden \parencite{leach_hypertext_2020}.} (HTTP) angewiesen ist. Dabei findet der Architekturstil, sowohl Anwendung für ganze Systeme, als auch in komplexen Anwendungen mit einer Vielzahl ein einzelnen Services.

%Todo Korrekturlesen von Yola

\subsubsection{Monolithe Architektur}

Der Begriff \textit{\enquote{Monolith}} leitet sich vom altgriechischen \textit{\enquote{monólithos}} ab und bedeutet \textit{\enquote{aus einem Stein}} \parencites[vlg.][]{duden_nodate}[vgl.][]{dwds_nodate}. In der Gesteinskunde wird da mit ein natürlich entstandener Gesteinsblock bezeichnet, der komplett aus einer Gesteinsart besteht \parencite[vgl.][]{dwds_nodate}.

Nach Rod Stephens liegt eine monolithische Softwarearchitektur vor, wenn jegliche Funktionalität des Systems miteinander verbunden ist. Dabei spricht er auf die Verbindung von Dateneingabe, Datenausgabe, Datenverarbeitung, sowie Fehlerhandhabung und Benutzeroberflächen \parencite[vgl.][S. 94]{stephens_beginning_2015}.

Anders sieht es Sam Newman. Ihm nach liegt ein Monolites System schon vor, wenn die gesamte Funktionalität eines Systems gemeinsam über ein Deployment-Prozess bereitgestellt wird \parencite[vgl.][Kap. 2.2]{newman_monolith_2019}. Somit muss nicht zwingend jegliche Logik miteinander verbunden sein. Er unterteilt Monolithe Systeme in drei Kategorien: Einzelprozess Monolithe, Modulare Monolithe und verteilte Monolithe \parencite[vgl.][Kap. 2.2]{newman_monolith_2019}.

Der Einzelprozess Monolith ist die gängigste Form und deckt sich mit der Definition von Rod Stephens. Somit handelt es sich dabei, um ein System bei dem das gesamte System ein Prozess abbildet. Dies bedeutet, dass jegliche Funktionalität auf einander aufbauend ist und nur eine Datenspeicherung für die gesamte Anwendung verwendet wird \parencite[vgl.][Kap. 2.2.1]{newman_monolith_2019}.
Anders ist dies beim Modularen System. Dieses zeichnet sich darin aus, dass die Funktionalität in einzelne Module geteilt wird und sogar einzelne Module eine separate Datenspeicherung besitzen können \parencite[vgl.][Kap. 2.2.2]{newman_monolith_2019}. Diese Form von monolithen System ist jedoch seltener und wird nur von einzelnen Unternehmen eingesetzt.
Im Gegensatz zu verteilten Systemen sind die einzelnen Komponenten nicht auf separaten Computern verteilt und werden durch einen Deployment-Prozess online gestellt. Des Weiteren sind die einzelnen Module nur leicht entkoppelt, so kann es immer noch Abhängigkeiten geben \parencite[vgl.][Kap. 2.2.2]{newman_monolith_2019}. Unterschiedlich davon sind verteilte Monolithe. Diese sind komplett entkoppelt und kommunizieren nur noch über definierte Schnittstellen \parencite[vlg.][S. 116]{starke_effektive_2015}. Sie erfüllen somit jegliche Anforderungen an ein verteiltes System, sind jedoch in einem einzigen Bereitstellungsprozess gebündelt. Diese From wird jedoch kaum verwendet, da sowohl Nachteile auf Grund der Verteilung, als auch durch das gemeinsame Bereitstellen, entstehen.

Weder Stephens, als auch Newman geben Vorgaben hinsichtlich der Gliederung innerhalb eines Monolithischen Systems. Demnach kann eine Model-View-Controller Ansatz als Monolithisches System gelten, solange es einheitlich deployed wird. Anders ist es mit einem verteilten System, da nach Definition ein Monolithen System kein verteiltes System sein kann. 

Im Rahmen dieser Arbeit wird bei jeglichen weiteren Referieren auf den Begriff Monolithen System stets von einem Einzelprozess Monolithen ausgegangen, außer es wird expliziert von einem Modularen, oder verteilten Monolithen geschrieben. Dadurch sollen beide Definitionen berücksichtig werden.

Im Vergleich zu einem verteilten System gibt es einige Vor- als auch Nachteile \parencite[vgl.][Kap. 2.2.4 und Kap. 2.2.5]{newman_monolith_2019}. So ist das bereitstellen eines Monolithen System einfacher, da es ein Bereitstellungsprozess für die gesamte Anwendung gibt. Wiederum führt dies dazu, dass der Prozess deutlich länger dauert. Diese Tatsache ist insbesondere gravierend, wenn vermehrt kleine Änderungen vorgenommen werden. Anderseits vereinfacht eine Anwendung, die als ein Prozess zu sehen ist, die Fehlersuche und ermöglicht es Funktionen mehrfach zu verwenden. So lassen sich Funktionen und Klassen in einem Einzelprozess Monolithen mehrfach verwenden und schneller neue Funktionen umsetzen. Jedoch verursacht dies, dass schnell Abhängigkeiten entstehen können und Änderungen ungewollte Fehler verursachen. Dadurch wird die Umsetzung von neuen Funktionen mit steigender Codemenge verlangsamt und der Einstieg von neuen Teammitgliedern erschwert.

Bei größeren Unternehmen mit mehreren Team kommt hinzu, dass es leicht zu Konflikten kommen kann, da alle auf die gleiche Codebase zugreifen. So führt ein Monolithen System dazu, dass bei vielen Entwickler viele Absprachen nötig sind und es zu Problemen bei der Zusammenführung von Funktionen kommen kann \parencite[vgl.][Kap. 2.2.4]{newman_monolith_2019}. Anders ist es beim Erstellen von System übergreifenden Test. Diese werden durch ein Monolithen System begünstigt und können im Vergleich zu einem verteilten System einfacher umgesetzt werden  \parencite[vgl.][Kap. 2.2.5]{newman_monolith_2019}.

\subsubsection{Einordnung des aktuellen Systems von PluraPolit}


\subsection{Microservices}

%	Todo Literatur finden und Abgrenzung schreiben. Es gab in dem Buch von Newman einen schnitt über die Geschichte und somit auch die Entwicklung des Begiffs.

%Todo der erste Satz muss noch einmal überprüft werden. Er kann so nicht richtig sein.
Der Begriff Microservices wird mehrdeutig verwendet. So wird je nach Perspektive entweder eine Softwarearchitektur oder eine Komponente einer solchen Architektur beschrieben. Eng verbunden mit diesem Begriff sind auch dynamische Systeme und Konzeptionierung von Unternehmensstrukturen. Demnach sieht Eberhard Wolf unter Microservices ein Modellierungskonzept, welches dazu dient größere Softwaresysteme in kleinere Einheiten zu teilen \parencite[vgl.][Kap. 1.1]{wolff_microservices_2018}. Dabei hat die Aufteilung Auswirkungen auf die Organisation als auch auf die Entwicklungsprozesse.

Nach Sam Newman ist ein Microservice ein \textit{\enquote{eigenständige ausführbare Softwarekomponente, die innerhalb eines Anwendungssystems mit anderen Softwarekomponenten kollaboriert}} \parencite[][Kap. 2.1]{newman_monolith_2019}. Sie zeichnet sich durch das kommunizieren über definierte Netzwerkschnittstellen aus und formt in Vereinigungen eine Microservices-Architektur. Ein Microservice umfasst dabei die Datenspeicherung, Datenverarbeitung und Datendarstellung und besitzt eine gut definierte Benutzeroberfläche \parencite[vgl.][Kap. 2.1]{newman_monolith_2019}.

Ergänzend dazu schreibt Wolf, dass sich das Konzept der Microservices-Architektur aus der Philosophie vom Unix Betriebssystem ableitet, welches nach Peter H. Salus folgende drei Leitpunkte umfasst \parencites{salus_quarter_1994}[vgl.][Kap. 1.1]{wolff_microservices_2018}:
\begin{itemize}
	\item Schreibe Programme, sodass sie nur eine Aufgabe erledigen und diese gut.
	\item Schreibe Programme, die zusammen arbeiten.
	\item Schreibe Programme, welche über definierte Schnittstellen (Textstream) kommunizieren.
\end{itemize}

Nach James Lewis sind Microservices kleine Anwendungen, die unabhängig bereitgestellt, getestet und skaliert werden. Ebenfalls wie Wolf beschreibt er diese Programme, als einfach zu verstehen, die nur eine Aufgabe übernehmen.

Zusammenfassend lässt sich sagen, dass der Begriff Microservices nicht einheitlich definiert ist und es sich zwei unterschiedliche Perspektiven ergeben. Zum einen beschreibt es ein Modellierungskonzept, welches Auswirkungen auf Unternehmenssturkur und Managemententscheidungen hat und zum anderen kennzeichnet es eine Software-Architektur, die in eigenständige Komponente geteilt ist.

In dieser Arbeit werden beiden Ansichten beleuchtet, um infolgedessen eine umfassende Einschätzung für PluraPolit zu geben.

\subsection{Microservices als Modellierungskonzept}

In diesem Abschnitt möchte ich mehr auf die Mircoservice-Architektur eingehen. Dabei möchte ich die Perspektive des verteilten Systems mehr beleuchten und die Unternehmsstruktur mehr vordern. Es wäre daher glaube sinnvoll den Abschnitt mit der Conways Law dem anzustellen oder einzu bauen.

\subsubsection{Conway's Law}

Die Abhängigkeit der Kommunikationsstruktur im Unternehmen zur verwendeten Software Architektur.

\begin{itemize}
	\item Begriff erklären
	\item Auswirkungen für die Teamstruktur beschreiben
\end{itemize}

\subsubsection{Domain Driven Design}

\begin{itemize}
	\item Was ist DDD?
	\item Was sind Bounded Contexts?
	\item 	Bespiele dazu.
	\item Schlussfolgerung für PluraPolit
\end{itemize}

\subsubsection{Cohesion und Coupling}

In diesem Abschnitt möchte ich diese beiden Begriffe erklären und noch einmal den Zusammenhang zwischen Funktionlität hervorheben. Es soll klar werden wie Microservices geteilt werden und was es bedeutet Logig hinter definierten Schnitstellen zu verstecken.

Abschließend zu diesem Abschnitt soll die dritte Bedingung für Microservices erstellt werden: \textbf{eine in teamsgeteilte, gemischte Unternehmensstruktur}

\subsection{Microservice}

In diesem Abschnitt möchte ich Microservice definieren und die einzelnen Merkmale benennen.
Es sollen zwei Eigenschaften deutlich als solche hervorstechen: Eigenständige Komponente und standatiesierte Kommunikation.

\subsubsection{Eigenständige Komponente}

In diesem Abschnitt möchte ich detalierter auf die Eigenschaft als eigenständige Komponente eingehen. Ich möchte beschreiben, welche Vorteile es für das einsetzen ergibt und was sich weitere Merkmale sich dadruch ergeben.

Benennen möchte ich dabei:
\begin{itemize}
	\item deployment
	\item technology
	\item max ein Team ist verantwortlich
	\item DB
	\item Feature bezogen (nach UNIX Philo)
	\item wenige Abhängigkeiten -> bleibende Produktivität
	\item leichter Einstieg
	\item bedarf angepasste skalierung
\end{itemize}

Zum Ende des Kapitels möchte ich darauf eingehen, dass dies auch mit sich führt, dass der Aufwand fürs Refaktoren insbesondere bei Veränderungen über mehrere Services hinweg aufwendig ist. Die Aufwand steigt im vergleich zu unverteilten Systemen. Es stellt sich somit als ein deutlicher Nachteil heraus Funktionalität später großflächig zu ändern.
Somit ist als Bedingung festzuhalten erst ein Produktmarketfit zu besitzen.

\subsection{Produktmarketfit bestimmen}

In diesem Abschnitt soll geklärt werden, wie ein Produktmarketfit bestimmt werden kann. Hierfür möchte ich viel aus Running Lean ziehen.

Es geht darum zum Einen Produktmarketfit zu definieren und zum anderen eine Methode zu gegeben an dieser ein solcher Zustand gemessen werden kann.

\subsection{Kommunikation von Services}

In diesem Abschnitt möchte ich darauf eingehen welche anforderungen eine entkoppelte Infrastuktur an das Netztwerk und an die Kommunikation über standards hat.

Ich möchte die unterschiedlichen Standards vorstellen und die zweite Anforderung für Microservices hinsichtlich des Netzwerks stellen.

Es sollen auf folgende Kommunikationsmethoden eingegangen werden.
\begin{itemize}
	\item Request / Response
	\item REST
	\item HTTP-Operationen (GET, PUT, PATCH, POST, DELETE)
	\item gRPC (habe selbst keine Ahnung was das ist)
\end{itemize}

\subsubsection{CAP - Theorem}

System, Konsistenz, Verfügbarkeit und Partitionstoleranz können in einem verteilten System nicht gleichzeitig erfüllt sein.
Die einzelnen Begriffe erklären \begin{itemize}
	\item Konsistenz
	\item Verfügbarkeit
	\item Partitionstoleranz
\end{itemize}

\subsection{Bedingungen ableiten}

Die Bedingungen aus den forhergehenden Kapiteln einführen und zusammenfügen.


\section{Methodik}
\section{Auswertung der Interviews und Anwendung auf PluraPolit}
\begin{itemize}
	\item Public subscribe pattern
	\item SOA 
\end{itemize}

\section{Fazit}

\newpage

% Literaturverzeichnis
\printbibliography[title=Literaturverzeichnis]

\newpage

% Abbildungsverzeichnis
\listoffigures

\newpage

\input{sections/eidesstattliche}

\end{document}